% Options for packages loaded elsewhere
\PassOptionsToPackage{unicode}{hyperref}
\PassOptionsToPackage{hyphens}{url}
%
\documentclass[
]{article}
\usepackage{amsmath,amssymb}
\usepackage{lmodern}
\usepackage{iftex}
\ifPDFTeX
  \usepackage[T1]{fontenc}
  \usepackage[utf8]{inputenc}
  \usepackage{textcomp} % provide euro and other symbols
\else % if luatex or xetex
  \usepackage{unicode-math}
  \defaultfontfeatures{Scale=MatchLowercase}
  \defaultfontfeatures[\rmfamily]{Ligatures=TeX,Scale=1}
\fi
% Use upquote if available, for straight quotes in verbatim environments
\IfFileExists{upquote.sty}{\usepackage{upquote}}{}
\IfFileExists{microtype.sty}{% use microtype if available
  \usepackage[]{microtype}
  \UseMicrotypeSet[protrusion]{basicmath} % disable protrusion for tt fonts
}{}
\makeatletter
\@ifundefined{KOMAClassName}{% if non-KOMA class
  \IfFileExists{parskip.sty}{%
    \usepackage{parskip}
  }{% else
    \setlength{\parindent}{0pt}
    \setlength{\parskip}{6pt plus 2pt minus 1pt}}
}{% if KOMA class
  \KOMAoptions{parskip=half}}
\makeatother
\usepackage{xcolor}
\IfFileExists{xurl.sty}{\usepackage{xurl}}{} % add URL line breaks if available
\IfFileExists{bookmark.sty}{\usepackage{bookmark}}{\usepackage{hyperref}}
\hypersetup{
  pdftitle={ Producing dynamic reports},
  hidelinks,
  pdfcreator={LaTeX via pandoc}}
\urlstyle{same} % disable monospaced font for URLs
\usepackage[margin=1in]{geometry}
\usepackage{graphicx}
\makeatletter
\def\maxwidth{\ifdim\Gin@nat@width>\linewidth\linewidth\else\Gin@nat@width\fi}
\def\maxheight{\ifdim\Gin@nat@height>\textheight\textheight\else\Gin@nat@height\fi}
\makeatother
% Scale images if necessary, so that they will not overflow the page
% margins by default, and it is still possible to overwrite the defaults
% using explicit options in \includegraphics[width, height, ...]{}
\setkeys{Gin}{width=\maxwidth,height=\maxheight,keepaspectratio}
% Set default figure placement to htbp
\makeatletter
\def\fps@figure{htbp}
\makeatother
\setlength{\emergencystretch}{3em} % prevent overfull lines
\providecommand{\tightlist}{%
  \setlength{\itemsep}{0pt}\setlength{\parskip}{0pt}}
\setcounter{secnumdepth}{-\maxdimen} % remove section numbering
\ifLuaTeX
  \usepackage{selnolig}  % disable illegal ligatures
\fi

\title{\includegraphics{./4cf3a1be231a55e00832f18f22d4480eeb96ac94.png}
Producing dynamic reports}
\usepackage{etoolbox}
\makeatletter
\providecommand{\subtitle}[1]{% add subtitle to \maketitle
  \apptocmd{\@title}{\par {\large #1 \par}}{}{}
}
\makeatother
\subtitle{Introduction}
\author{}
\date{\vspace{-2.5em}2022-06-17 07:44:37}

\begin{document}
\maketitle

{
\setcounter{tocdepth}{2}
\tableofcontents
}
\hypertarget{about-this-resource}{%
\section{About this resource}\label{about-this-resource}}

\hypertarget{introduction}{%
\subsection{Introduction}\label{introduction}}

Dynamic reports are data-driven documents that can be periodically
updated as their underlying data changes.

This resource is a brief introduction to writing dynamic reports using
\href{https://www.r-project.org/}{R},
\href{http://rmarkdown.rstudio.com}{Rmarkdown}, and
\href{https://rstudio.cloud}{RStudio Cloud}. It is a two hour
interactive session designed for those without previous programming
experience who work in health and care. It gives a general introduction
to the tools, and then some useful examples of frequently-met
data-driven tasks. It is intended as a starting point for automating
your report writing processes, with the aim of replacing, enhancing, or
simplifying, manual report writing. The demonstration also covers ways
of producing the report in a variety of formats including .pdf, .docx
(Word format), and .html (webpage).

Some examples of routine reports that might be targets for re-working in
a dynamic format:

\begin{itemize}
\tightlist
\item
  service-use reports (admission figures, bed utilisation)
\item
  annual reports (public health annual reports)
\item
  engagement and impact data (training, outreach)
\end{itemize}

\hypertarget{who-is-this-demo-for}{%
\subsubsection{Who is this demo for?}\label{who-is-this-demo-for}}

This learning resource is intended for someone who:

\begin{itemize}
\tightlist
\item
  is currently spending lots of time manually updating reports in
  health, care, or housing
\item
  \textbf{and} would like to reduce the time spent on this routine work
  over the medium-term (months)
\item
  \textbf{and} has some time to spend reworking this report-writing
  process
\item
  \textbf{and} are open to gaining some basic programming experience
\end{itemize}

Note too that it is a learning resource. You'll need some time to read
this introduction and get set for the training session, as well as for
the two hour training session itself.

\hypertarget{what-this-demo-is-not}{%
\subsubsection{What this demo is not}\label{what-this-demo-is-not}}

\begin{itemize}
\tightlist
\item
  it's not a full introduction to working in
  \href{https://www.r-project.org/}{R}/\href{http://rmarkdown.rstudio.com}{Rmarkdown}.
  \href{https://www.r-project.org/}{R} is deep (like most analytic
  platforms), and this demonstration covers only a little bit of the
  available functionality
\item
  it's not an introduction to everything that you might like to do with
  a data-driven report either. Instead, it covers some simple examples
  of common tasks as a way of familiarising you with a different
  workflow
\item
  it's definitely not a production-ready replacement for your existing
  reports. It's a learning resource, rather than a pre-packed
  replacement
\item
  it won't teach you how to make real-time dashboards. If you need your
  report to update more than about once a day, you should consider
  building a dashboard instead
  (\href{https://powerbi.microsoft.com/en-gb/}{Power BI},
  \href{https://www.r-project.org/}{R}/\href{https://shiny.rstudio.com/}{Shiny},
  or \href{https://plotly.com/dash/}{Dash})
\end{itemize}

\hypertarget{about-dynamic-reports}{%
\subsection{About dynamic reports}\label{about-dynamic-reports}}

\hypertarget{is-it-worth-it-for-me}{%
\subsubsection{Is it worth it for me?}\label{is-it-worth-it-for-me}}

\begin{figure}
\centering
\includegraphics{./3ac468ff4b8a0a3b056d5968032984459d92624d.png}
\caption{A recent personal example of a copy and paste error encountered
while writing a short report. This would have been avoided if I'd been
using a dynamic report.}
\end{figure}

We can't provide general guidance about whether automating some specific
report will be worth the investment of time for you. But, we can suggest
some strengths and weaknesses of dynamic reports compared to
traditionally-produced static reports that might be helpful if you're
thinking about automating some of your work:

\hypertarget{strengths}{%
\paragraph{Strengths}\label{strengths}}

\begin{itemize}
\tightlist
\item
  \textbf{fidelity}. A dynamic report means no manual updating of text,
  graphs, and figures. No more copy and paste problems (see the Teams
  chat message above)
\item
  \textbf{standardisation}. If you have lots of similar reports to
  write, this dynamic approach simplifies the process of making your
  reports look alike by applying a standard approach to formatting,
  analysis, and so on. This is a huge time-saver, especially true if you
  have lots of graphs to produce.
\item
  \textbf{reproducibility}. If someone else looks at the source code for
  your report, they can see exactly how your figures are produced. That
  means that, when it's time to hand over production of your report to a
  colleague, you won't need to explain how to build the report. It's all
  contained in the source-code.
\item
  \textbf{efficiency}. Again assuming that everything works properly,
  updating a dynamic report with new data is much quicker than updating
  a comparable report by hand.
\end{itemize}

\hypertarget{weaknesses}{%
\paragraph{Weaknesses}\label{weaknesses}}

\begin{itemize}
\tightlist
\item
  \textbf{set-up cost}. Particularly if you are new to this kind of
  work, writing a new dynamic report takes much longer than a
  traditional report. There's plenty to learn, and it can be hard to
  find that time needed to learn, and to rework an established process.
\item
  \textbf{errors}. While dynamic reports is a great way of avoiding
  minor errors from copying and pasting, it can introduce entirely new
  sources of error from reworking your analysis. Reports, particularly
  early on, need frequent careful checking to ensure that you haven't
  swapped small frequent errors for large, subtle ones.
\item
  \textbf{Information Governance and data protection concerns} about the
  platforms used to write these kinds of reports.
\item
  \textbf{novelty}. Changing processes is often controversial, with a
  degree of resistance. For example, there's a strong lock-in to the
  Microsoft Office suite (Word and Excel especially) for report writing.
  Suggesting something new can be daunting. Change management and QI
  methods might be helpful if you think this is likely to be a
  substantial concern for your project.
\end{itemize}

On balance, we think that dynamic reports have more advantages than
disadvantages. But decisions about whether to automate a report will
depend on the local factors in play. So that's something that I'd be
happy to discuss during (or after) the session.

\hypertarget{getting-started}{%
\subsection{Getting started}\label{getting-started}}

\hypertarget{how-does-it-work}{%
\subsubsection{How does it work}\label{how-does-it-work}}

We'll use three tools to write the report. First, we'll use the web
service \href{https://rstudio.cloud}{RStudio Cloud}. This allows us to
run \href{https://www.r-project.org/}{R} without installing any software
or making any changes to our computer. Next, we'll use the markup
language \href{http://rmarkdown.rstudio.com}{Rmarkdown} to add some text
and images to our report. Finally, we'll use the
\href{https://www.r-project.org/}{R} programming language to do some
simple data handling, analysis, and visualisation.

Just in case you'd like to look into how these tools work in advance of
the session, we would recommend:

\begin{itemize}
\tightlist
\item
  an introductory \href{https://www.r-project.org/}{R} webinar -
  \textbf{will link to John MacKintosh webinar recording here}
\item
  a quick
  \href{https://www.r-bloggers.com/2022/02/rstudio-cloud-how-to-get-started-for-free/}{introduction
  to RStudio Cloud from R bloggers}
\item
  \href{https://rmarkdown.rstudio.com/authoring_quick_tour.html}{Rstudio's
  quick tour of Rmarkdown} is a great place to start if this is all new
  to you
\item
  If you're already somewhat familiar with working in
  \href{https://www.r-project.org/}{R}, you might prefer to start with
  the \href{https://r4ds.had.co.nz/r-markdown.html}{Rmarkdown chapter in
  the \emph{R for Data Science} book}
\end{itemize}

\href{https://rstudio.cloud}{RStudio Cloud} is easy to set-up and free
for small-scale work like this demo. This makes it by far the easiest
way to get going from scratch if you've never worked with
\href{https://www.r-project.org/}{R} before. Note that because it's a
web service, it requires you to upload your data to their servers, which
might makes it unsuitable for production work in health and care owing
to information governance concerns. That said, it's easy to tranfer
projects from \href{https://rstudio.cloud}{RStudio Cloud} to an
installed version of \href{https://www.r-project.org/}{R}, so don't
worry that what you learn here will be tied to the cloud forever.

My suggestion would be not to try and use this demo to change your way
of writing reports under pressure. There's quite a lot to think about
here, and you might need to spend a good bit of time working out how to
adapt this demonstration to fit your report. Think of this as the start
of a journey, rather than a destination.

\hypertarget{joining-instructions}{%
\subsubsection{Joining instructions}\label{joining-instructions}}

You'll need to do a little bit of preparation before the training
session. I'd advise you to leave about 15 minutes to do this, so that we
can make a prompt start to the session. If you're new to
\href{https://rstudio.cloud}{RStudio Cloud}, please follow the
step-by-step instructions below. If you're more experienced in this kind
of work, you short-cut by signing-in to
\href{https://rstudio.cloud/}{RStudio Cloud}, creating a new project
from the
\href{https://unito.io/blog/guide-to-github-for-project-managers/}{GitHub}
Repository at \url{https://github.com/bclarke-nes/Dynamic-report-demo},
and then opening the demo.Rmd file. Otherwise:

\begin{enumerate}
\def\labelenumi{\arabic{enumi}.}
\item
  Go to \url{https://rstudio.cloud/}
\item
  If you have an account, log in. Otherwise, create an account by
  selecting \texttt{Get\ started\ for\ free}, follow the steps, and then
  sign-in
\end{enumerate}

\includegraphics[width=300px]{https://i.imgur.com/2p1w6ts}

\begin{enumerate}
\def\labelenumi{\arabic{enumi}.}
\setcounter{enumi}{2}
\tightlist
\item
  Add a new project on \href{https://rstudio.cloud}{RStudio Cloud} by
  clicking
  \texttt{New\ Project\ \textgreater{}\textgreater{}\ New\ Project\ from\ a\ Git\ repository}.
  When prompted, enter the URL
  \texttt{https://github.com/bclarke-nes/Dynamic-report-demo}
\end{enumerate}

\includegraphics{./7dbeeebb17d27037b8865ac5a7e5a5a41d056a29.png}

\begin{enumerate}
\def\labelenumi{\arabic{enumi}.}
\setcounter{enumi}{3}
\tightlist
\item
  That will give you a new project containing the files needed for this
  demonstration:
\end{enumerate}

\includegraphics{./deccfa3234d755eea1f7eee40119eec81e26eb2c.png}

\begin{enumerate}
\def\labelenumi{\arabic{enumi}.}
\setcounter{enumi}{4}
\tightlist
\item
  Open the demo.Rmd file from the Files pane
\item
  A word of reassurance. If you haven't done any coding before, your
  first sight of the \href{https://rstudio.cloud}{RStudio Cloud}
  interface might be a shock. Don't worry about this - we'll go through
  some tips at the start of the session for setting up RStudio to make
  it more friendly.
\end{enumerate}

\hypertarget{aims-and-objectives}{%
\subsection{Aims and objectives}\label{aims-and-objectives}}

This session will:

\begin{itemize}
\tightlist
\item
  Give an introduction to why we should write dynamic reports
\item
  What kinds of report are most suitable to automate
\item
  Give a basic overview of
  \href{https://www.r-project.org/}{R}/\href{http://rmarkdown.rstudio.com}{Rmarkdown}/\href{https://rstudio.cloud}{RStudio
  Cloud}, including basic data handling, simple data analysis, and
  drawing graphs
\item
  Show how these functions can be integrated into a simple report format
  that will update as the underlying data changes
\end{itemize}

By the end of this session, the user should:

\begin{itemize}
\tightlist
\item
  Have gained a basic understanding of
  \href{https://www.r-project.org/}{R}, and how
  \href{https://rstudio.cloud}{RStudio Cloud} and
  \href{http://rmarkdown.rstudio.com}{Rmarkdown} can be used to generate
  reports
\item
  Be familiar with the advantages and disadvantages of working in this
  way compared to traditional manual report writing in Word and Excel
\item
  Be able to confidently navigate the parts of an
  \href{http://rmarkdown.rstudio.com}{Rmarkdown} document
\item
  Be able to recognise some simple \href{https://www.r-project.org/}{R}
  code, and with suitable assistance interpret it
\item
  Be able to seek suitable help for their R code needs
\item
  Produce and tweak simple descriptive statistical measures, and simple
  visualisations in \href{https://www.r-project.org/}{R}
\item
  Be confident adding dynamic text elements to an
  \href{http://rmarkdown.rstudio.com}{Rmarkdown} document
\end{itemize}

\end{document}
